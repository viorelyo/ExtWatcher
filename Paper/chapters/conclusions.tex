\chapter{Conclusion and future work}
\label{chapter:conclusion}

The main objectives behind the proposed approach for creating a Cloud application, which could scan PDF documents and return a verdict was successfully fulfilled. This paper demonstrates the capabilities of Machine Learning algorithms applied on cybersecurity tasks, which in some cases can provide even better results than a classic antivirus application. In addition to the Machine Learning framework responsible for PDF scanning, we also provide an automated solution for file uploading, which can take an according action to the scanned file based on it's result, in order to assure the safety of user's computer. The symbiosis of these two creates a complex system that distributes the performance-expensive task of malware analysis from an endpoint machine to a remote server. This way the user should not worry that his computer will be compromised by any PDF document, which these days is one of the most exchanged file formats. \par 
The next step in extending our created product will be making the framework publicly available, by deploying it on a Cloud service. Currently the automatic file uploader is supported only on Windows OS and the users have the alternative to manually submit the URL of a PDF file in order to scan it. In the future we plan to develop a similar background application for automating file upload for Android OS and possibly other operating systems. The Cloud application is also included in our upgrades plan. It's architecture already permits the integration of other Machine Learning models for classification of new file formats. The purpose of the upcoming research will be Microsoft Office Document formats (DOC, XLS, PPT), as well as PE (Portable Executable) and DLL (Dynamic-link library) files. \par
We think that now is the era for transfering all of the performance draining applications to the Cloud.