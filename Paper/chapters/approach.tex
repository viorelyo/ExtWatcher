\chapter{Proposed approach}
\label{chapter:proposedApproach}

For solving the problem of detecting malicious PDF documents, we came with a Machine Learning based approach. We took into account the importance of keeping the documents content confidentiality, as well as the need for ensuring the speed of analysis process. This solution will be next integrated into a Cloud framework responsible for remote documents scanning.

\section{Dataset}
\label{section:dataset}
The first step in building the optimal \textit{ML} model was to create a dataset which would train the classifier. Our target was to collect malicious and benign PDF samples used in real life scenarios. Circa 70\% of the clean and malicious documents were downloaded from the online malware repository \textit{Contagio} \cite{contagio}, which provides samples collected from various open sources. Nevertheless, for more recent malicious PDFs we used VirusTotal \cite{virustotal}, Hybrid Analysis \cite{hybridanalysis} and VirSCAN \cite{virscan}, that allow searching files by their hash (\textit{MD5}, \textit{SHA1}, \textit{SHA256}). Some of them provide access to the entire file collection so that we can order them by upload date and search for the most recent uploaded harmful files. Many clean samples were collected from public sources using Google Search Operators, i.e., \code{filetype:pdf}, for restricting search results of PDF files. Additionally, we completed the dataset with more than 100 clean interactive PDF files, that contain embedded 3D Widgets, incorporated JavaScript games etc. These clean samples should train the ML model, so as to correctly classify files even in corner cases. 

\begin{table}[H]
	\caption{Dataset of PDF samples}
	\label{table:pdfSamples}
        \centering
            \begin{tabular}{p{2.5cm} p{9.5cm}}
                \toprule
                
				\textbf{Category} & \textbf{Source} \\
				\hline 
                \texttt{benign} & Contagio, Google Queries, Tetra4D, PdfScripting \\
                \hline
				\texttt{malicious} & Contagio, VirusTotal, Hybrid Analysis, VirSCAN \\
                
                \bottomrule
			\end{tabular}
\end{table}

\section{Proof of Concept}
\label{section:poc}

In the following subsections we will touch upon algorithms and techniques we have used to achieve the purpose of this thesis, namely detecting malicious PDF documents using a Machine Learning based solution. Figure \ref{mlsteps} represents the entire process which was followed while developing the \textit{ML} model. Each process phase will be fathomed in the following.

\begin{figure}[H]
	\centerline{\includegraphics[scale=0.5]{figures/ml.png}}  
	\caption{Machine Learning model creation}
	\label{mlsteps}
\end{figure}

\subsection{Feature Selection}
As we already know, Machine Learning is based on mathematical concepts, hence the algorithms consist of performing mathematical operations to identify patterns in data. In order to use our dataset as input for ML algorithms, first of all we need to bring the data in an appropiate form. Therefor the most relevant data from a PDF document should be extracted and transformed into a vectorized form. As already seen in Table \ref{table:pdfentries}, there are several standard PDF entries which can be used for malicious intentions. For featurizing PDF documents we have used \textit{PDFiD} from \textit{PDF Tools} suite \cite{pdftools}, which is a Python script that selects 22 features from a PDF file, including keywords commonly found in malicious documents, name obfuscations etc. The output of the script is a list of PDF entries and their occurences (see Figure \ref{pdfidoutput}). Before turning this featurized form of the PDF file into input for ML algorithms, the data should be normalized to [0, 1] range. This is an important step in order to avoid ML issues when features are on drastically different scales (see \cite{mlCookbook}). The vectorized form of the PDF document is built by applying the \textit{Min-Max Normalization} (\ref{eq:1}) strategy on the array of features extracted by \textit{PDFiD}.

\begin{equation}
	\label{eq:1}
	z = \frac{x - min(x)}{max(x) - min(x)}
\end{equation}

\begin{figure}[H]
	\centerline{\includegraphics[scale=0.6]{figures/pdfidoutput.png}}  
	\caption{Example of results of \textit{PDFiD} applied on a benign PDF file (left) and on a malicious one (right)}
	\label{pdfidoutput}
\end{figure}

\subsection{Classification Techniques}
\subsection{Performance Evaluation}
\subsection{Experiment}     % Metasploit + Kali

\section{Used technologies}
\label{section:technologies}
\subsection{Microsoft .NET Framework}
\subsection{PDF Tools and Metasploit Framework}
\cite{zeltser}
\subsection{Python Flask}
\subsection{Scikit-learn Machine Learning Library}
\subsection{ReactJS Framework}
