\chapter{Scientific Problem}
\label{section:scientificProblem}

\section{Problem definition}
\label{section:problemDefinition}
The purpose of the research in this thesis consists in approaching some alternative methods to replace classic file scanning techniques. Particularly, our target is to demonstrate that Machine Learning algorithms can be used as an efficient mechanism for malicios PDF files detection. Additionally, we want to present a framework that can remotely analyze suspicious PDF files by applying earlier mentioned performant algorithms. This framework has several advantages: 
\begin{itemize}
    \item Takes on the complicated task of applying classification model for uploaded files and giving the analysis result to the users.
    \item It is designed to be deployed as a Cloud Application on a powerfull server that can handle multiple requests at the same time, lacking users care about having high specifications for their computers.
    \item Guarantees the privacy of the scanned documents. The algorithms doesn't require the documents content in order to decide whether they are malicios or benign. Also, after analysis process, none of the documents are stored on the Cloud.
    \item Keeps the detection algorithms up-to-date. It won't be the users responsability anymore to regularly update the software antivirus installed on their computers. 
    \item Provides a generic implementation for analyzing uploaded files. The support for a new file extension to be scanned, can be effortlessly added by adding a classifier Machine Learning model for the wanted file format.
\end{itemize}
Integrating both Machine Learning and Cloud Computing into Cybersecurity should provide a significant progress to this field.


\section{Background processes in Microsoft Windows}
\label{section:backgroundProc}

\section{Filesystem monitoring}
\label{section:filesystem}

\section{Analyzing PDF File Structure}
\label{section:pdfStructure}

\section{Malware in PDF}
\label{section:malwareInPDF}
% Since PDF documents became a well known solution for information sharing, they also became a good target for cybercriminals.

\section{Machine Learning for Malware Detection}
\label{section:mlForMalware}

\section{Benefits of Cloud Computing}
\label{section:cloudComputing}
% Using a cloud-based anti-virus makes it much easier to manage, especially as there's no need to constantly update software on multiple devices. However, it is worth bearing in mind that installing anti-virus to combat malware can only be one part of an overall IT security policy.
% https://www.redhat.com/en/topics/cloud-native-apps/what-are-cloud-applications
% Simply put, a cloud application is software that users access primarily through the internet, meaning at least some of it is managed by a server, not the user’s local machine. This basic definition doesn’t fully describe how cloud applications have reshaped markets and business models, though. If designed well, cloud applications can offer a user experience like a program installed entirely on a local machine, but with reduced resource needs, more convenient updating, and the ability to access functionality across different devices.
% Nowadays with cloud use being common, it's more efficient to install anti-virus software on your network.
% benfetis - low performant devices + mobile
% http://lxiao.xmu.edu.cn/Papers/Mobile%20Offloading%20for%20Cloud-based%20Malware%20Detections%20with%20Learning.pdf
%Advantages:
% Fast computation to run more advanced and complex detection algorithms
% More accurate detection with a large-size signature database
% Address zero-day vulnerabilitie
% Cloud-based malware detection vs. local detection
%  Transmission delay, computation speed, detection accuracy, storage cost
%  User competition vs. cooperation in the malware detection
%  Compete for the limited network bandwidth
%  Contribute the malware signature database to improve the malware
%  detection accuracy at the cloud
%Cloud computation resource=1Gbps
% Trace generation speed=1Mbps
% Transmission cost factor=0.2
% Accuracy coefficient=0.5 