\chapter{Introduction}
\label{chapter:introduction}

\section{Context}
\label{section:context} 
The informational technology progress, that is in continous growth, brings a lot of benefits along with new responsibilities. There are plenty of applications that we use everyday across the entire World Wide Web. We don't even realize how much of our personal information is transfered to the virtual environment. That being said, it's important to be prepared for cybersecurity threats that can misuse our sensitive data. The problem is that the cyber attackers develop a lot of \textit{hacking}\footnote{attempt to gain unauthorized access to data in a system} techniques, which are becoming increasingly difficult to detect. The popular software applications are the best target to inject malicious behavior. This happened with the Adobe PDF format. PDF documents are well known, trustworthy files and they are a global solution for sharing information. There are a lot of PDF readers and even browsers and email applications have support to open these files for viewing. It became so convenient to work with this format, that users interact with PDF documents without noticing any possible danger. However, PDF is a often used attack vector. The large number of discovered PDF vulnerabilities and also the support of embedding Javascript code into documents are just some of the most exploited methods.


\section{Motivation}
\label{section:motivation}
Under the guise of seeming harmless, PDF documents are used on a daily basis across numerous public institutions, private companies and for personal purposes. Most of the people don't consider that these files could be dangerous and just copy the documents to their computers and access them. It is enough for one PDF to be malicious and the entire network affiliated to an institution can become compromised. The cyber attackers succeed in achieving their goals, but the victim institution requires huge resources, both financially and time wise, to restore the integrity and security of their infrastructure. A measure to combat the described situation is having a real time protection installed on the computer. The most common solution, antivirus applications, work by signature matching, which is effective for detecting previously identified malware\footnote{any software intentionally designed to cause damage to a computer, server, client, or computer network}. This means that all the antivirus applications require permanent updates in order to keep their malware databases up-to-date. Many users opt out of security solutions because of the multitude of hardware requirments they need. In this thesis we will go through a new approach that implies transfering complex detection algorithms from user's computers to a remote service, whose only task is analyzing suspicious uploaded PDF documents. Consequently the computers that will use this solution get rid of additional workload while still maintaining the system secure.


\section{Paper structure and original contributions}
\label{section:structure}
The main contribution of this thesis is the research, design and implementation of alternative security solution oriented on multiplatform use and performance efficiency. For achieving this, the work was split into three important parts:
\begin{enumerate}
    \item \textbf{Real time monitoring system on Windows}, whose goal is to notify the user when a specific file type, in our case PDF, is downloaded. The main focus when designing this software was to keep its functionality basic, so that it would require minimal computer's resources. As soon as it detects a new such file, it submits it to the Cloud for further analysis.
    \item \textbf{Development of an intelligent algorithm for PDF classification} implies studying the PDF format skeleton, researching popular attack vectors using PDFs, extracting the most relevant features from a document and feeding them to the most fit Machine Learning classification algorithm. A part of this effort was also spent for searching the dataset, based on which, the classification model was trained to correctly detect malicious PDF files.
    \item \textbf{Cloud based application for remote scanning}, which is a generic application that can integrate models such as the developed one for PDF classification and can use them for analyzing the uploaded files. It offers an user friendly dashboard for visualizing the scanned files and also provides the possibility of submitting an URL containing a PDF file. The application will care about downloading and analyzing the file, showing the user the scanning results. This is a perfect option for users that require the Cloud features from a smartphone.
\end{enumerate}

The remaining parts of this thesis are structured in the following manner: Chapter 2 contains the research made in the past years on malicious PDFs and Cloud Computing as an antimalware solution. Chapter 3 introduces the reader into fundamental mechanisms used for implementing the final application, as well as a brief history of malware in PDF. In Chapter 4 we present a Proof of Concept for developed classification model based on recreating a pseudo attack using malicious PDF. The 5th Chapter is putting each component together and presents the overall design of the application. Chapter 6 is meant for our final conclusions and we also show some of the future ideas for improving the application.

