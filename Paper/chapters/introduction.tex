\chapter{Introduction}
\label{chapter:introduction}

\section{Context}
\label{section:context} 
The informational technology progress, that is in continous growth, brings a lot of benefits along with new responsabilities. There are plenty of applications that we use everyday across the entire World Wide Web. We don't even realize how much of our personal information is transfered to the virtual environment. That being said, it's important to be prepared for cybersecurity threats that can misuse our sensitive data. The problem is that the cyber attackers develop a lot of \textit{hacking}\footnote{attempt to gain unauthorized access to data in a system} techniques, which are becoming increasingly difficult to detect. The popular software applications are the best target to inject malicious behavior. This happened with the Adobe PDF format. PDF documents are well known, trustworthy files and they are a global solution for sharing information. There are a lot of PDF readers and even browsers and email applications have support to open these files for viewing. It became so convenient to work with this format, that users interact with PDF documents without noticing any possible danger. However PDF is a often used attack vector. The large number of discovered PDF vulnerabilities and also the support of embedding Javascript code into documents are just some of the most exploited methods.


\section{Motivation}
\label{section:motivation}
Under the guise of seeming harmless, PDF documents are used on a daily basis across numerous public institutions, private companies and for personal purposes. Most of the people don't consider that these files could be dangerous and just copy the documents to their computers and access them. It is enough for one PDF to be malicious and the entire network affiliated to an institution becomes compromised. The cyber attackers succeed in achieving their goals, but the victim institution requires huge resources, both financially and time wise, to restore the integrity and security of their infrastructure. A measure to combat the described situation is having an active real time protection installed on the computer. The most common solution, antivirus applications, work by signature matching, which is effective for detecting previously identified malware\footnote{any software intentionally designed to cause damage to a computer, server, client, or computer network}. This means that all the antivirus applications require permanent updates in order to keep their malware databases up-to-date.




\section{Paper structure and original contributions}
\label{section:structure}

